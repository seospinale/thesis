\documentclass[10pt]{article}  

%%%%%%%% PREÁMBULO %%%%%%%%%%%%
\title{Propuesta grado}
\usepackage[spanish]{babel} %Indica que escribiermos en español
\usepackage[utf8]{inputenc} %Indica qué codificación se está usando ISO-8859-1(latin1)  o utf8  
\usepackage{amsmath} % Comandos extras para matemáticas (cajas para ecuaciones,
% etc)
\usepackage{amssymb} % Simbolos matematicos (por lo tanto)
\usepackage{graphicx} % Incluir imágenes en LaTeX
\usepackage{color} % Para colorear texto
\usepackage{subfigure} % subfiguras
\usepackage{float} %Podemos usar el especificador [H] en las figuras para que se
% queden donde queramos
\usepackage{capt-of} % Permite usar etiquetas fuera de elementos flotantes
% (etiquetas de figuras)
\usepackage{sidecap} % Para poner el texto de las imágenes al lado
	\sidecaptionvpos{figure}{c} % Para que el texto se alinie al centro vertical
\usepackage{caption} % Para poder quitar numeracion de figuras
\usepackage{commath} % funcionalidades extras para diferenciales, integrales,
% etc (\od, \dif, etc)
\usepackage{graphicx, amsmath, amsthm, latexsym, amssymb, amsfonts, epsfig, float, enumerate, color, listings,  graphicx, fancyhdr}

\usepackage{cancel} % para cancelar expresiones (\cancelto{0}{x})
\usepackage[table,xcdraw]{xcolor} %Agregar color a los cuadros
\usepackage{anysize} 					% Para personalizar el ancho de  los márgenes

\marginsize{2cm}{2cm}{2cm}{2cm} % Izquierda, derecha, arriba, abajo

\usepackage{appendix}
\renewcommand{\appendixname}{Apéndices}
\renewcommand{\appendixtocname}{Apéndices}
\renewcommand{\appendixpagename}{Apéndices} 

% Para que las referencias sean hipervínculos a las figuras o ecuaciones y
% aparezcan en color
\usepackage[colorlinks=true,plainpages=true,citecolor=blue,linkcolor=blue]{hyperref}
%\usepackage{hyperref} 
% Para agregar encabezado y pie de página
\usepackage{fancyhdr} 
\pagestyle{fancy}
\fancyhf{}
\fancyhead[L]{\footnotesize Universidad Nacional de Colombia} %encabezado izquierda
\fancyhead[R]{\footnotesize Sede Medellín}   % dereecha
\fancyfoot[R]{\footnotesize Sebastián Ospina Leal}  % Pie derecha
\fancyfoot[C]{\thepage}  % centro
\fancyfoot[L]{\footnotesize Propuesta trabajo de grado}  %izquierda
\renewcommand{\footrulewidth}{0.4pt}


\usepackage{listings} % Para usar código fuente
\definecolor{dkgreen}{rgb}{0,0.6,0} % Definimos colores para usar en el código
\definecolor{gray}{rgb}{0.5,0.5,0.5} 
% configuración para el lenguaje que queramos utilizar
\lstset{language=Python,
   keywords={break,case,catch,continue,else,elif,end,for,function,
      global,if,otherwise,persistent,return,switch,try,while},
   basicstyle=\ttfamily,
   keywordstyle=\color{blue},
   commentstyle=\color{red},
   stringstyle=\color{dkgreen},
   numbers=left,
   numberstyle=\tiny\color{gray},
   stepnumber=1,
   numbersep=10pt,
   backgroundcolor=\color{white},
   tabsize=4,
   showspaces=false,
   showstringspaces=false}

\newcommand{\sen}{\operatorname{\sen}}	% Definimos el comando \sen para el seno
%en español

\title{Propuesta de Proyecto de Grado}

%%%%%%%% TERMINA PREÁMBULO %%%%%%%%%%%%

\begin{document}

%%%%%%%%%%%%%%%%%%%%%%%%%%%%%%%%%% PORTADA %%%%%%%%%%%%%%%%%%%%%%%%%%%%%%%%%%%%%%%%%%%%
																					%%%
\begin{center}																		%%%
\newcommand{\HRule}{\rule{\linewidth}{0.5mm}}									%%%\left
 																					%%%
\begin{minipage}{0.48\textwidth} \begin{flushleft}
\includegraphics[scale = 0.4]{EscudoU.jpg}
\end{flushleft}\end{minipage}
\begin{minipage}{0.48\textwidth} \begin{flushright}
% \includegraphics[scale = 0.45]{Sesqui.jpg}
\end{flushright}\end{minipage}

													 								%%%
\vspace*{-1.5cm}								%%%
																					%%%	
\textsc{\huge Universidad Nacional\\ \vspace{5px} de Colombia}\\[1.5cm]	

\textsc{\large PROPUESTA DE TRABAJO DE GRADO PARA OPTAR POR EL TITULO DE MAESTRIA EN INGENIERÏA DE LOS RECURSOS HIDRAULICOS}\\[2cm]													%%%

%%%
    																				%%%
 			\vspace*{1cm}																		%%%
																					%%%
\HRule \\[0.4cm]																	%%%
{ \huge \bfseries Title about floods}\\[0.4cm]	%%%
 																					%%%
\HRule \\[1.5cm]																	%%%
 																				%%%
																					%%%
\begin{minipage}{0.46\textwidth}													%%%
\begin{flushleft} \large															%%%
\emph{Autor:}\\	
Sebastián Ospina Leal\\
C.c: 1037646272\\ 
Email: seospinale@unal.edu.co
%%%
			%\vspace*{2cm}	
            													%%%
										 						%%%
\end{flushleft}																		%%%
\end{minipage}		
																%%%
\begin{minipage}{0.52\textwidth}		
\vspace{-0.6cm}											%%%
\begin{flushright} \large															%%%
\emph{Director:} \\																	%%%
Carlos David Hoyos Ortíz\\
 	\\
\emph{Co-director:} \\																	%%%
Nicolas Velasquez Girón\\
 \\
												%%%
\end{flushright}																	%%%
\end{minipage}	
\vspace*{1cm}
%\begin{flushleft}
 	
%\end{flushleft}
%%%
 		\flushleft{\textbf{\Large Facultad de Minas}	}\\																		%%%
\vspace{2cm} 																				
\begin{center}																					
{\large \today}																	%%%
 			\end{center}												  						
\end{center}							 											
																					
\newpage																		
%%%%%%%%%%%%%%%%%%%% TERMINA PORTADA %%%%%%%%%%%%%%%%%%%%%%%%%%%%%%%%

\tableofcontents 

\newpage

% \section{Proponentes.}

% \begin{tabular}{cc}
% Sebastián Ospina Leal & \parbox[t]{20cm}{Estudiante de Maestría en Ingeniería – Recursos Hidráulicos}\\
% Carlos David Hoyos Ortíz & \parbox[t]{20cm}{Director }\\
% Nicolas Velasquez Girón & \parbox[t]{20cm}{Director }
% \end{tabular}


\include{document/problem}

\section{Objectives}

\subsection{General objective}

\subsection{Specific objectives}

\begin{itemize}
    \item 1
\end{itemize}

\include{document/literature}

\include{document/methodology}

% \include{document/}





\begin{thebibliography}{References}

\bibitem{u} Universidad Nacional de Colombia.
\end{thebibliography}  

% \newpage
% \section{Firma.}

% \chapter{ \textbf{Firma del proponente }}
% \\[20pt]


% \makebox[2.5in]{\hrulefill} 

% Nombres y Apellidos\\ 

% \chapter{  \textbf{Firma del director }}
% \\[20pt]

% \makebox[2.5in]{\hrulefill} 

% Nombres y Apellidos\\







\end{document}